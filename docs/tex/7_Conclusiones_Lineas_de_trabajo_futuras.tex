\capitulo{7}{Conclusiones y Líneas de trabajo futuras}
En este punto de la memoria, se tratan las conclusiones obtenidas de la realización de este proyecto, junto con las posibles opciones a añadir que no se han implementado a fecha de la presentación de este proyecto y que quedan abiertas a continuidad.

\section{Conclusiones}
Las conclusiones obtenidas del desarrollo del proyecto son las siguientes:
\begin{itemize}
    \item El objetivo principal del proyecto, crear una interfaz de Dialogflow más amigable que la que ofrece Google, se ha cumplido satisfactoriamente. Los usuarios que quieran gestionar los chatbots creados con Dialogflow, podrán hacerlo de forma eficiente y sencilla.
    \item Al usar un lenguaje de programación tan extendido como es Python, ha permitido tener una gran diversidad de opciones, tanto en su desarrollo, como a la hora de elegir entre infinidad de librerías que le otorgan una gran flexibilidad al proyecto.
    \item El desarrollo de este proyecto ha permitido utilizar la mayor parte de los conocimientos estudiados en la carrera, además de incrementar los relacionados con desarrollo software, diseño de interfaces o procesadores del lenguaje, entre otros.
    \item Gracias a la investigación realizada sobre los chatbots, se ha podido comprender la inteligencia artificial y los procesadores de lenguaje, así como determinar la forma en la que ambos campos interactúan para ofrecer a los usuarios este tipo de tecnologías que facilitan muchas tareas.
    \item El uso de metodologías ágiles durante todo el proyecto, ha permitido mantener una organización en todas las tareas a completar, estableciendo objetivos en periodos cortos de tiempo, haciendo mucho más eficiente, flexible y sencillo cumplirlos.
    \item Durante todo este trabajo, se han usado varias metodologías y herramientas que han fomentado la calidad y el correcto funcionamiento del proyecto. Añadir, que algunas de ellas han llegado a crear sobrecarga, tanto funcional, como de usabilidad de las mismas, pero es cierto, que todas ellas han permitido que el proyecto haya salido adelante.
\end{itemize}

Como conclusión personal, este proyecto ha cumplido con las expectativas y los objetivos personales que se habían establecido inicialmente. No ha sido un camino sencillo, pero ha llevado a obtener un resultado digno del esfuerzo entregado.

\section{Líneas de Trabajo Futuras}
En cada proyecto que se crea y más hablando de un desarrollo software, quedan muchos frentes abiertos por los que se podría continuar, ya que siempre es posible añadir nuevas funcionalidades, mejoras o extensiones. A continuación, se muestran algunas de estas ideas:
\begin{itemize}
    \item Migración de ChatScriptor a otra plataformas como aplicación de escritorio o aplicación móvil, entre otras.
    \item Añadir más modelos de traducción estudiando mecanismos para que no se sobrecargue la aplicación.
    \item Poder obtener un nuevo servicio que pueda almacenar mucha más información y permita que sea mucho más eficiente que el actual.
    \item Para facilitar el almacenamiento de información, desarrollar una funcionalidad que permita añadir una ``fecha de caducidad'' para que cuando un chatbot lleve almacenado mucho tiempo y no se haya usado, este se elimine, enviando una copia del chatbot al correo del propietario e informándole de la eliminación del mismo del servidor.
    \item Teniendo los datos de los números de frases de entrenamiento, entidades y sus respectivas respuestas, añadir un informe estadístico con los datos de todos ellos.
    \item Añadir filtros de ordenamiento en las páginas de selección de chatbots, entidades e intents que faciliten la organización de los mismos.
    \item Realizar, mediante \textit{web scraping}, la obtención del historial desde Dialogflow, permitiendo realizar un informe estadístico de los usos de los agentes.
\end{itemize}