\capitulo{1}{Introducción}
Los chatbots son herramientas software que nos permiten hacer consultas y obtener una respuesta de forma rápida. Se desarrollan mediante inteligencia artificial, ya que algunos de ellos, son capaces de interpretar imágenes o sonidos. Los más sencillos, actúan siguiendo una serie de guiones que le son establecidos por su creador.

Actualmente, este tipo de interacciones hombre-máquina, se utilizan con fines comerciales, industriales y educativos. Permiten tener acceso a la información sin tener que realizar interacciones directas y poder resolver cualquier pregunta las 24 horas del día, los 365 días del año. Esto proporciona un gran número de ventajas, tanto para el usuario que crea el chatbot como para quién hace uso del mismo.

Existen gran variedad de empresas que permiten la creación, mantenimiento y seguimiento de los chatbots tanto de manera gratuita como de pago. Algunas de las más conocidas y usadas son: Microsoft Bot Framework, Amazon Lex, IBM Watson Assistant, Salesforce Einstein Bots, SAP Conversational AI y Google Dialogflow.

Con la explosión de las inteligencias artificiales actuales, se está desarrollando aún más software que permiten crear chatbots. Debido a las necesidades actuales, muchos de los usuarios de este tipo de herramientas no tienen conocimientos de programación, con lo que es importante tener opciones que estén basadas en  programación \textit{low code}\footnote{La programación \textit{low code} hace referencia a la capacidad de desarrollar software reduciendo el uso de programación tradicional \cite{¿Quéesun25:online}.}.

Nos encontramos con varios tipos de chatbots, entre ellos están los asistentes que utilizan algoritmos de generación de lenguaje natural basados en modelos avanzados como las redes neuronales recurrentes (RNR). Estos permiten crear respuestas y mantener conversaciones fluidas, ya que tienen la capacidad de originar sus propias respuestas usando los conceptos relevantes de la discusión, adaptándose al usuario. Aunque no todo son ventajas, ya que sus respuestas pueden ser incompletas o erróneas, debido, comúnmente, a la falta de entrenamiento de estos modelos.

Por otro lado, existen los chatbots basados en guiones, patrones o reglas predefinidas. Estos chatbots utilizan estas pautas para conversar con el usuario, dándole las respuestas según estas instrucciones, haciendo responsable al creador de añadir suficientes oraciones a las diferentes entradas. Es por esto, que poseen una flexibilidad bastante baja así como poca capacidad de interpretación de las conversaciones, pero al revés que los anteriores, las respuestas serán las correctas siempre y cuando coincidan con los patrones. Estos chatbots se usan en campos estructurados que requieren una mayor precisión de respuesta, como puede ser un asistente de preguntas frecuentes, con información general.

Este trabajo se centra en Google Dialogflow, la plataforma que nos ofrece Google mediante Google Cloud. Los productos que se pueden crear con esta aplicación son del segundo tipo, es decir, chatbots basados en guiones. El motivo de esta elección es la capacidad que tiene para expandirse y desarrollarse, ya que al no poseer una compleja técnica de creación de chatbots, permite añadir funcionalidades y mejoras. Además de que es un software gratuito en su versión básica.

A lo largo de este proyecto, se desarrolla una aplicación web basada en la oficial, realizando un previo análisis para encontrar puntos a mejorar, que permitan al usuario tener una mejor experiencia de uso.


\section{Estructura de la memoria}\label{estructura-de-la-memoria}
La memoria posee la siguiente estructura:
	\begin{itemize}
		\tightlist		
		\item
			\textbf{Introducción: }
			descripción del problema y cuál es la solución elegida, estructura de la memoria, de los anexos y los materiales adjuntados.		
		\item
			\textbf{Objetivos del proyecto: }
			explicación de forma precisa y concisa de los objetivos que se persiguen con la realización del proyecto.		
		\item
			\textbf{Conceptos teóricos: }
			descripción de los conceptos teóricos estudiados y aplicados a la solución del proyecto.		
		\item
			\textbf{Técnicas y herramientas: }
			descripción de las técnicas y herramientas usadas tanto para la gestión como para el desarrollo del proyecto.		
		\item
			\textbf{Aspectos relevantes del desarrollo: }
			exposición de los puntos destacables que han surgido durante la realización del proyecto.			
		\item
			\textbf{Trabajos relacionados: }
			relación del proyecto actual con otros de similares características.			
		\item
			\textbf{Conclusiones y líneas de trabajo futuras: }
			conclusiones del trabajo y las diferentes posibles mejoras y ampliaciones del proyecto.
	\end{itemize}

\section{Estructura de los anexos}\label{estructura-de-los-anexos}
Los anexos poseen la siguiente estructura:
	\begin{itemize}
		\tightlist		
		\item
			\textbf{Plan del proyecto software: }
			planificación y viabilidad del proyecto.		
		\item
			\textbf{Especificación de requisitos del software: }
			fase de análisis, describiendo los requisitos y objetivos del software.		
		\item
			\textbf{Especificación de diseño: }
			descripción de diseño de datos, diseño procedimental y diseño arquitectónico.		
		\item
			\textbf{Manual del programador: }
			documentación relacionada con la estructura, instalación, ejecución, etc. 
		\item
			\textbf{Manual del usuario: }
			guía de usuario.	
	\end{itemize}

\section{Materiales}\label{materiales-adjuntos}
Materiales adjuntados junto con la memoria:
	\begin{itemize}
		\tightlist		
            \item 
                Página web desplegada: \url{https://chatscriptor.azurewebsites.net/}
            \item
                Repositorio del proyecto: \url{https://github.com/clv1003/Chat-Scriptor}
            \item Chatbots (Agentes) de ejemplo: \\
            \url{https://universidaddeburgos-my.sharepoint.com/:f:/g/personal/clv1003_alu_ubu_es/EpCnl0HnKBRDi2I4tRlaYTwBvYpzJZDTckz0qWBcacVkzw?e=OSJIFG}
	\end{itemize}