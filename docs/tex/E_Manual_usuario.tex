\apendice{Documentación de usuario}
\section{Introducción}
En este anexo. se desarrolla toda la documentación necesaria para los usuarios de ChatScriptor, describiendo los requisitos de usuario, el proceso de instalación y el manual de usuario. Todo ello, para hacer conocedor al navegante de todo lo que puede hacer y cómo hacerlo.

\section{Requisitos de usuarios}
En esta sección, se determinan los requisitos que el usuario debe cumplir para poder utilizar la aplicación web.
\begin{itemize}
    \item Dispositivo: es necesario que sea compatible con la navegación web, como puede ser un ordenador, \textit{smarthphone} o \textit{tablet}, aunque se recomienda el uso de un ordenador.
    \item Sistema operativo: no hay limitación en cuanto a sistema operativo ya que se trata de una aplicación multiplataforna.
    \item Conexión a internet: para poder usar la aplicación web, es necesario el acceso a una conexión a internet.
    \item Navegador web: cualquiera mientras esté actualizado. Algunos ejemplos son \textit{Google Chrome}, \textit{Microsoft Edge} o \textit{Mozilla Firefox}, entre otros.
    \item Requisitos adicionales: conocimientos básicos de navegación web y cuenta de correo electrónico.
\end{itemize}

\section{Instalación}
ChatScriptor es una aplicación web, por lo que no necesita ningún tipo de instalación, más allá de tener en el dispositivo un navegador web.

Se recomienda usar un ordenador con uno de los siguientes navegadores, aunque, tal y como se ha indicado en los requisitos, se puede usar otro tipo de dispositivo y otros navegadores:
\begin{itemize}
    \item \textit{Google Chrome}: \url{https://www.google.com/intl/es_es/chrome/}
    \item \textit{Microsoft Edge}: \url{https://www.microsoft.com/es-es/edge/download?form=MA13FJ}
    \item \textit{Mozilla Firefox}: \url{https://www.mozilla.org/es-ES/firefox/new/}
    \item \textit{Opera}: \url{https://www.opera.com/es/download}
\end{itemize}

Una vez se tienen los requisitos, solo será necesario acceder a la aplicación web: \url{https://chatscriptor.azurewebsites.net/}

\section{Manual del usuario}
En este manual, se mostrará todo el proceso para realizar las acciones disponibles en ChatScriptor.

\subsection{Importación y exportación desde Dialogflow}
El primer paso para poder usar esta aplicación, es tener en el sistema un chatbot previamente creado en Dialogflow.

Para ello, se accederá a Dialogflow (\url{https://dialogflow.cloud.google.com/?hl=es-419}) y se creará un chatbot siguiendo los pasos que te indica la interfaz.

Una vez está creado, se entra a la configuración del agente y, en el apartado \textit{Export and Import}, se exporta el archivo \textit{ZIP} del chatbot.
\imagen{exportar_chatbot}{Exportación de agente desde Dialogflow}{1}

La realización del proceso inverso, es decir, importar el agente en Dialogflow, se hace desde esta misma pantalla.

\textbf{Advertencia}: el funcionamiento de las opciones de \textit{RESTORE} y de \textit{IMPORT} no siempre funciona bien, pero esto no es problema de ChatScriptor, sino de Dialogflow.

\subsection{Registro de nuevo usuario}
Para poder usar ChatScriptor, es necesario tener una cuenta de usuario registrada en el servidor.

\begin{enumerate}
    \item Acceder a la página: \url{https://chatscriptor.azurewebsites.net/}.
    \item Desde la página de \textit{login}, acceder al registro de usuarios o mediante la dirección: \url{https://chatscriptor.azurewebsites.net/register}.
    \item Se introducirán los datos del usuario: nombre de usuario, correo electrónico y una contraseña.
    \imagen{MU_Registro}{Registro de un nuevo usuario}{1}
    \item Una vez se complete el registro, la página redirigirá al inicio de sesión.
    \item Se introducirá el correo electrónico y la contraseña.
    \imagen{MU_Login}{Inicio de sesión del usuario recien creado}{1}
\end{enumerate}

Con esto, se dispondrá de una cuenta habilitada para el uso de ChatScriptor.

\subsection{Importación y exportación de agente}
Una vez se accede a la página principal, se verá que no tenemos ningún agente añadido. Para añadirlo, se accederá desde la barra superior a \textit{Importar/Exportar}.
\imagen{MU_PaginaPrincipal}{Pagina principal y acceso a la importación del nuevo agente}{1}

Cuando completemos el acceso a esta página, será posible seleccionar el archivo \textit{ZIP} que se quiere importar. Al terminar, aparecerá un mensaje, indicando que el archivo se ha importado en el sistema.
\imagen{MU_Importar}{Proceso de importación de un nuevo agente}{1}

Para obtener el agente (exportarlo), en la misma pantalla, se encuentra una opción de exportación de agentes, donde es posible indicar el que se quiere adquirir.
\imagen{MU_Exportacion}{Localización de la exportación de archivos}{1}

\subsection{Datos del agente}
Ahora que está disponible el chatbot, desde la página de inicio se accederá a este agente. A continuación, aparecerá un menú, donde se elegirá el bloque al que se quiera acceder (agente, entidades, \textit{intents}).
\imagen{MU_Menu}{Menú del agente actual}{1}

Una vez se entra a cualquiera de los bloques, cuando se visualice la información, se podrá añadir, modificar o eliminar en aquellos puntos donde se indique en la interfaz.

\textbf{Importante}: si se va a eliminar algún parámetro, tener en cuenta que este no se podrá recuperar.

\subsubsection{Agente}
Si desde el menú anterior, se accede a \textit{Agente}, se cargará la información del agente en pantalla.
\imagen{CP 7.1 - 1}{Interfaz con la información del agente}{1}
Como se ha indicado anteriormente, en cada formulario se podrá modificar los diferentes parámetros.

\subsubsection{Entidades}
Si desde el menú anterior, se accede a \textit{Entidades}, se cargarán todas las entidades disponibles para el chatbot, mostrándose en una lista de botones con las entidades.
\imagen{MU_Entidades}{Interfaz con el listado de entidades}{1}

\textbf{Entidad} \\
Entrando a una de estas entidades, se accede a su información. Como se ha especificado anteriormente, los formularios permiten la modificación y los botones rojos con ``X'' permiten su eliminación (\textbf{recordatorio}, si se elimina, el dato eliminado no se podrá recuperar).
\imagen{CP 7.2 - 1}{Interfaz con la información de la entidad}{1}
\subsubsection{Intents}
Si desde el menú anterior, se accede a \textit{Intents}, se cargarán todos los \textit{intents} disponibles para el chatbot y, al igual que con las entidades, se mostrará una lista de botones con los \textit{intents}.
\imagen{MU_Intents}{Interfaz con el listado de \textit{intents}}{1}

Como se puede observar en la \textit{Figura E.10}, existe un botón con el nombre de ``Informe''. Al pulsar dicho botón, se genera una nueva pantalla que devuelve un informe con las diferentes frases de entrenamiento y respuestas. Además, es posible pulsar en el nombre de cada uno de los \textit{intents} que aparecen, lo que nos dará acceso directo al \textit{intent}.
\imagen{MU_Informe}{Interfaz con el informe de los \textit{intents}}{1}

\textbf{Intent} \\
Entrando a un de estos \textit{intents}, se accederá a su información. Al igual que en el resto de casos, es posible modificar los parámetros y eliminar algunos de ellos (\textbf{recordatorio}, si se elimina, el dato eliminado no se podrá recuperar).
\imagen{CP 7.3 - 1}{Interfaz con la información del \textit{intent}}{1}

\subsection{Buscadores}
Al igual que cualquier otro buscador, los formularios disponibles en numerosa cantidad de pantallas a lo largo de ChatScript, permiten obtener la información reducida, encontrando el termino introducido.

La diferencia se encuentra en que el funcionamiento de los buscadores dependerá de la pantalla en la que nos encontremos.
\begin{itemize}
    \item Buscador en la página de inicio: devolverá cualquier coincidencia que encuentre en todos los chatbots que tengamos disponibles.
    \item Buscador en la página general de un agente: devolverá cualquier coincidencia que encuentre entre los datos del asistente (agente, entidades e \textit{intents}).
    \item Buscador en la página de datos del agente: devolverá cualquier coincidencia que encuentre en el agente.
    \item Buscador en la página de entidades: devolverá cualquier coincidencia que encuentre entre todas las entidades.
    \item Buscador en la página de \textit{intents}: devolverá cualquier coincidencia que encuentre entre todos los \textit{intents}.
    \item Buscador en la página de una entidad: devolverá cualquier coincidencia que encuentre en esa entidad.
    \item Buscador en la página de un \textit{intent}: devolverá cualquier coincidencia que encuentre en ese \textit{intent}.
\end{itemize}

\subsection{Traductor}
El traductor se encuentra disponible en la pantalla de los datos del agente (Figura E.7). Actualmente, está solo disponible la traducción de español a inglés y viceversa.

El proceso de traducción es lento, por lo que no es de extrañar que tarde varios minutos en traducir el agente completo.
\imagen{MU_Traductor}{Traducción de un chatbot}{1}


\subsection{Otras indicaciones}
\subsubsection{Cerrar sesión}
Para cerrar la sesión actual, se encuentra disponible en todas la pantalla, un botón en la parte superior derecha, que permite realizar esta acción.

\subsubsection{Acerca de}
En la barra de navegación superior, se encuentra un apartado de información sobre el proyecto, acceso al repositorio del código fuente y el correo electrónico de contacto.
